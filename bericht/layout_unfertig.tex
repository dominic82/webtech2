\section{Layout}

\subsection{Einleitung}
Das Layout eines Webprojektes ist das erste, was dem Benutzer bei Benutzung eines solchen auff�llt. Bei der Planung des Layouts wurde darum auf mehrere Aspekte eingegangen. Es soll schnell laden, �bersichtlich, einfach zu bedienen und zentral ver�nderbar sein w�hrend es 
optisch ansprechend sein soll. In den folgenden Abschnitten geht es um die L�sungsans�tze, die wir zu den obigen Herausforderungen gew�hlt haben.

\subsection{Ladezeit}
Lange Wartezeiten zwischen einem Request und der Response resultieren oft daher, dass viele gr��ere Bilder geladen werden m�ssen. Um das m�glichst zu verhindern sind nur wenige Grafiken
mit einer begrenzten Farbpalette (reduziert Dateigr��e) in das Grundlayout eingeflossen. Enthalten sind ein eigenes Logo und die Logos der Universit�t und der Informatikfakult�t, so wie ein Bild einer jubelnden Menge, welches zur Begr��ung angezeigt wird. 

\subsection{�bersichtlichkeit}

\subsection{Bedienbarkeit (Usability)}

\subsection{Ver�nderbarkeit}
In dem Modul Webtechnologien 1 wurden im Zusammenhang mit ver�nderlichem Layout, die Cascading Style Sheets (css) propagiert, so dass sie f�r uns erste Wahl waren. Es wurden, unabh�ngig von den dynamischen Inhalten, HTML-Datein angelegt, die aus <div> - Elemente und darin enthaltenen statischen Text (Platzhalter f�r dynamische Inhalte) bestehen. Diesen <div>s wurden Klassen und Ids zugewiesen, welche dann zentral von einer layout.css ihr optisches Erscheinungsbild bekommen haben. Auf die Verwendung von Inline-definierten Styles wurde gr��tenteils verzichtet, da dadurch die M�glichkeit einer zentralen Ver�nderung des Styles abhanden kommen w�rde. Aus den gut kommentierten HTML-Datein konnte dann jeder Ersteller einer Komponente/Page den, f�r ihn relevanten Teil extrahieren und daraus seine projektbezogene TML erzeugen. Positive Nebeneffekte der erzeugten HTML waren, dass klar definiert war, wie das Projekt mal aussehen w�rde und dass Dinge deutlich wurden, die bei der urspr�nglichen Planung vergessen und/oder weniger gut durchdacht worden sind.
  
BILD: first.jpg
(erstes Layout � HTML mit statischen Inhalten per css optisch angepasst)

\subsection{Optische\Haptische Wahrnehmung}
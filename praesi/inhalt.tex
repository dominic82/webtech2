\section{Live-Demo}

\subsection*{}
\begin{frame}{Live-Demo}
\begin{itemize}
\item Account erstellen
\item Account bearbeiten
\item Fan einladen
\item Fan werden
\item Shouts erstellen
\end{itemize}
\end{frame}

\section{Pagination}
\subsection*{}
\begin{frame}{Pagination}
Pagination bei Tapestry nur über Grid-Komponente \\
\begin{itemize}
\item Darstellung als Tabelle $\rightarrow$ lässt sich schlecht designen
\item mögliche Lösung: DOM per JavaScript verändern 
\end{itemize}
\pause
Ansonsten keine fertige Lösung! $\rightarrow$ Also: selber bauen... \\
\pause
\begin{itemize}
\item Erweiterung der Loop-Komponente
\item zusätzliche Parameter: aktuelle Seite, Anzahl Items pro Seite
\item Format-Parameter: \texttt{first,previous,pages,next,last,bottom,top}
\end{itemize}
\end{frame}

\section{Testen und Selenium}
\subsection*{}
\begin{frame}{Testen und Selenium}
unsere Test-Strategie:
\begin{itemize}
\item keine JUnit-Tests genutzt (nur für DAOs sinnvoll?)
\item Integrations-Tests im Tapestry-Kontext $\rightarrow$ per Hand! :-(
\end{itemize}
\pause
Testen mit Selenium:
\begin{itemize}
\item Integrations-Tests werden schneller und wiederholbar! 
\item können "`nebenher"' per Hand angelegt werden
\item müssen aber auch per Hand angepasst werden $\rightarrow$ aufwendig! :-(
\end{itemize}
\end{frame}

\section{LearnLib}
\subsection*{}
\begin{frame}{LearnLib}
unsere Erwartung:
\begin{itemize}
\item Nutzung vorhandener modellbasierter Selenium-Tests
\item automatischer Vergleich zwischen Software und Spezifikation
\item übersichtliches Modell der Software
\end{itemize}
\pause
die Realität sieht anders aus:
\begin{itemize}
\item Alphabet muss gut überlegt sein
\item nur ein "`triviales"' Modell erzeugt: Zwei-User-Login
\item "`interessante"' Abläufe in der Form nicht darstellbar! \\ $\rightarrow$ benötigt Gegenbeispiele!
\end{itemize}
\end{frame}

\section{Feature-Ausblick}
\subsection*{}
\begin{frame}{Feature-Ausblick}
\begin{itemize}
\item Backend (Administration) inkl. Rechtesystem
\item mehr JavaScript
	\begin{itemize}
	\item Anzeige neuer Shouts, Einladungen, Fans/Idols
	\item Pagination oder dynamisches Nachladen
	\end{itemize}
\pause	
\item Suche erweitern: Auto-Completion, Nachrichten, Hash-Tags
\item Organisation der Idols in Gruppen
\item Recommender-System für Idols	
\end{itemize}
\end{frame}

